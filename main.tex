\documentclass{beamer}
\usepackage[utf8]{inputenc}
\usetheme{cern}

%The CERN logo is legally protected. Please visit http://cern.ch/copyright for information on the terms of use of CERN content, including the CERN logo.

% The optional `\author` command defines the author and is displayed in the slide produced by the `\titlepage` command.
\author{Author Name}

% The optional `\title` command defines the title and is displayed in the slide produced by the `\titlepage` command.
\title{CERN Presentation title}

% The optional `\subtitle` command will add a smaller title below the main one, and will not be displayed in any of the slides' footer.
\subtitle{CERN Presentation subtitle}

% The optional `\date` command will display a custom free text date on the all of the slides' footer. If omitted today's date will be used.
%\date{Monday, 1st January 2018}

\begin{document}

\frontcover

% The optional `\titlepage` command will create a slide with the presentation's title, subtitle and author.
\frame{\titlepage}

% The optional `\tableofcontents` command will automatically create a table of contents based pm the sections.
\frame{\tableofcontents}

\section{Introduction}

\begin{frame}{Introduction}
This template is based on Jérôme Belleman's \texttt{beamer-cern} project. Fore more information please refer to: \small{\url{https://github.com/jeromebelleman/beamer-cern}}
\end{frame}

\section{Main message \#1}
\frame{\sectionpage}

\begin{frame}{Slide title}
A slide with a title and some content
\end{frame}

\begin{frame}
Another slide without a title and some content
\end{frame}

\section{Main message \#2}
\frame{\sectionpage}

\begin{frame}{Slide title}
	\begin{itemize}
		\item A slide
		\item with a title
		\item and some bullets
	\end{itemize}
\end{frame}

\section{Conclusion}

\begin{frame}{Conclusion}
	\begin{itemize}
		\item A slide
		\item with a title
		\item and some bullets
	\end{itemize}
\end{frame}

\backcover

\end{document}